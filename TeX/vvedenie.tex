\newpage
\begin{center}
	\textbf{ВВЕДЕНИЕ}
\end{center}
\addcontentsline{toc}{section}{ВВЕДЕНИЕ}

Данная работа относится к области компьютерного зрения.

Компьютерное зрение — это научное направление в области искусственного интеллекта, в частности робототехники, и связанные с ним технологии получения изображений объектов реального мира, их обработки и использования полученных данных для решения разного рода прикладных задач без участия (полного или частичного) человека.

Современные задачи компьютерного зрения разделяют на четыре вида, им однозначно не сопоставлены русскоязычные термины, поэтому во избежание неточностей представим исходную классификацию на английском языке:

\begin{itemize}
	\item Classification (далее классификация) — классификация изображения по типу объекта, которое оно содержит;
	\item Semantic segmentation (далее сегментация) — определение всех пикселей объектов определённого класса или фона на изображении. Если несколько объектов одного класса перекрываются, их пиксели никак не отделяются друг от друга;
	\item Object detection (далее локализация) — обнаружение всех объектов указанных классов и определение охватывающей рамки для каждого из них;
	\item Instance segmentation — определение пикселей, принадлежащих каждому объекту каждого класса по отдельности;
\end{itemize}

В данной работе исследованию подлежат задачи классификации и локализации объектов на изображении, так как они наиболее актуальны для робототехнических систем. Определение положения предметов в пространстве необходимо как для навигации мобильного робота, так и для позиционирования манипулятора на конвейере. 

За последнее десятилетие в связи с наращиванием вычислительных мощностей и доступностью большого объёма данных стало активно развиваться глубокое обучение, сейчас каждый может собрать в интернете набор данных и обучить на нём свою нейронную сеть, каждый день появляются новые модели, поэтому актуальной проблемой становится не разработка новой архитектуры, а нахождение ранее созданной модели, подходящей для поставленной задачи компьютерного зрения.

