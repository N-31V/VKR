\newpage
\begin{center}
	\textbf{ЗАКЛЮЧЕНИЕ}
\end{center}
\addcontentsline{toc}{section}{ЗАКЛЮЧЕНИЕ}
В ходе работы было проведено исследование архитектур нейронных сетей для классификации и локализации объектов на изображении: 
\begin{enumerate*}
	\item Рассмотрены существующие решения для классификации объектов, среди которых для дальнейшего сравнения были выбраны: ResNet-18, ResNet-50, ResNeXt-101-32x8d.
	\item Рассмотрены существующие решения для локализации объектов, среди которых для дальнейшего сравнения были выбраны Faster R-CNN и RetinaNet.
	\item Произведено изучение принципов работы выбранных архитектур.
	\item Произведено экспериментальное сравнение выбранных архитектур.
\end{enumerate*}

В результате сравнения моделей для классификации получены следующие рекомендации для использования: использовать ResNet-18 в задачах, требующих минимальной временной задержки или в условиях минимизации занимаемой программой памяти; использовать ResNeXt-101-32x8d для достижения максимальной точности в задачах, не требовательных к скорости распознавания,  в остальных случаях рекомендуется использовать ResNet-50, как компромиссный вариант, обеспечивающий достаточно высокую точность и скорость, приемлемую для распознавания в реальном времени.

Сравнение Faster R-CNN и RetinaNet показало, что первая архитектура превосходит вторую как по качеству работы, так и по скорости распознавания. Наиболее подходящая область для использования RetinaNet, учитывая особенность архитектуры, которая позволяет получать хорошо обобщённые признаки для маленьких объектов, -- это распознавание на изображениях, снятых с воздуха.  
